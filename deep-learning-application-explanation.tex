

\section{How is Deep Learning applied}
Attack individual key bytes.
Each Neural Network has 256 outputs, one for each key hypothesis.
The output itself is between 1 and 0, being the probability of the connected hypo being the key.

Training
In: labelled data
Out: weights

Attack
In: weights
In: traces
Out: values

Hi Fabrizio,

ich habe dir bisher noch keine genauere Erklärung zu den Fragen gegeben.

Sowohl in der Trainingsphase als auch in der Angriffsphase werden die einzelnen Keybytes individuell angegriffen.
Generell ist es bei den verwendeten NNs in der Trainingsphase so:
Input:
Die Anzahl die Neuronen des ersten Layers sind gleich der Anzahl der Samples in dem verwendeten Trace.
Pro Keybyte werden 1000 traces zum trainieren verwendet, die nacheinander eingespeist werden.

Output: 
Der letzte Layer des NNs hat dann 256 Neuronen, ein Neuron pro Keybyte Hypothese.
Der Output ansich ist letztendlich eine Zahl zwischen 1 und 0, welche die Wahrscheinlichkeit darstellt dass der zugehörige Byte der korrekte Keybyte ist.
Da der Schlüssel bekannt ist wird dann das Delta zwischen predicted und actual result errechnet und damit der Backpropagation algorithmus gestartet um die weights/feature maps zu anzupassen.


In der Angriffsphase läuft es wie folgt ab:
Input: 
Die vorher errechneten weights/feature maps werden wieder eingesetzt, das NN ist damit trainiert.
Pro angegriffenem Keybyte wird nun ein Trace in das NN eingespeist und die Wahrscheinlichkeiten pro Keyhypothese gespeichert. Das wird 20.000 mal wiederholt. Anschließend wird die Maximum Likeliness errechnet, woraus dann die richtigen Keybytes geschätzt werden und anschließend zu dem Schlüssel zusammen gesetzt werden können.

